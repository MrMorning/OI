\newproblem{10E. Greedy Change}
\begin{prob}
	给定一个货币系统,求最小的
	$w$,使得贪心(从高价值货
	币开始能选就选)使用的货币不是最优解(即个数不是最少的)。
	$n \le 400$。
\end{prob}

\begin{sol}
	专门有一篇论文针对这题
	《A Polynomial-time Algorithm for the Change-Making Problem》,
	下面仅是对该论文的简要重述。
	大致思想是找到$O(n^2)$个可能出现反例的$w$,然后每个$O(n)$验证。\par
	首先是几个定义。
	\begin{definition}
		定义一个\textbf{货币系统}为$n$维向量$C = (c_1, c_2, \cdots, c_n)$。
		定义$x$在该货币系统下的\textbf{表达}为$V = (v_1, v_2, \cdots, v_n)$,
		即$V \cdot C = x$。定义$|V|$为$V$用的货币个数。
	\end{definition}
	
	\begin{definition}
		$x$的\textbf{贪心表达}$G(x)$为$x$的表达中字典序最大的。
	\end{definition}
	显然,$x < y \Rightarrow G(x) < G(y)$。
	
	\begin{definition}
		$x$的\textbf{最小表达}$M(x)$为最小大小的表达中字典序最大的。
	\end{definition}
	于是,目标变成了找到最小的$w$,使$M(w) \not = G(w)$。
	
	\begin{theorem}
		称$U$是\textbf{贪心}的,当$U = G(U \cdot C)$;
		是\textbf{最小}的,当$U = M(U \cdot C)$。那么:
		\begin{enumerate}
			\item 如果$U \subset V$,且$V$是贪心的,那么$U$是贪心的。
			\item 如果$U \subset V$,且$V$是最小的,那么$U$是最小的。
		\end{enumerate}
	\end{theorem}
	
	\begin{proof}
		注意到向量加法对字典序的保序性:
		\begin{displaymath}
			A \le B \Leftrightarrow A + D \le B + D
		\end{displaymath}
		设$U'$为$U \cdot C$的任意一个表达。于是
		\begin{eqnarray}
		U' \cdot C &=& U \cdot C \nonumber \\
		(V - U + U') \cdot C &=& V \cdot C \nonumber \\
		V - U + U' &\le& V \nonumber \\
		U' &\le& U \nonumber
		\end{eqnarray}
		对于最小的证明,只需将$\le$替换成$\sqsubseteq$即可。
	\end{proof}
	
	下面开始构造可能的反例集合。设$w$为最小的使$G(w) \not = M(w)$的
	最小正整数。
	一个重要的结论是$G(w)$和$M(w)$中非0项的交集为空。否则可以同时
	减去某个$c_i$仍然满足$G(w) \not = M(w)$,故$w$不是最小的了。
	\par
	令$M(w) = (m_1, m_2, \cdots, m_n)$,$i,j$为第一个和最后一个
	非0项的下标。由于$M(w) < G(w)$,因此$G(w)$第$i$项为0,前
	$i-1$项中有一项非0。
	\begin{theorem}
		$M(w)$与$G(c_{i-1}-1)$相比,第$1$到$j-1$项相同,
		第$j$项大$1$,其余项全为0。
	\end{theorem}
	\begin{proof}
		因为$G(w)$前$i-1$项有一个非0,故$w \le c_{i-1}$。
		另一方面,将$M(w)$第$j$项减1得到$w-c_j$的最小且贪心
		表达。对比$M(w-c_j)=G(w-c_j)$与$M(w)$,可以发现前$i-1$项
		都为0,故$w-c_j < c_{i-1}$。
		\par
		令$V = (v_1, v_2, \cdots, v_n) = G(c_{i-1}-1)$。由于
		$c_{i-1} - 1 \ge c_i, v_i \not = 0$,可以将$v_i$和$m_i$
		同时减1得到$G(c_{i-1}-1-c_i)$和$G(w-c_i)$。有前面的
		不等式可知,$G(c_{i-1}-1-c_i) < G(w-c_i)$,将$i$位置
		的1加回去不影响相对字典序。故$V < M(w)$。\par
		另外,若将$m_j$减1,得到$G(w-c_j)$。因为$w-c_j \le c_{i-1}-1$,
		有$G(w-c_j) \le V$。综合可得$G(w-c_j) \le V < M(w)$。
		而$G(w-c_j)$与$M(w)$仅在$j$位置上差1,故$V$只能
		在$j$位置之后与$M(w)$不同。又因为$m_{j+1},m_{j+2}, \cdots, m_n$
		全为0,$V$只能在$j$位置比$M(w)$小。而$G(w-c_j) \le V$,
		所以$m_j - 1 \le v_j$。因此,$m_j$只能取$v_j+1$。
	\end{proof}
	通过枚举$i,j$,就能算得$M(w)$,以及$w$。再$O(n)$求出$G(w)$。
	总复杂度是$O(n^3)$。
\end{sol}
