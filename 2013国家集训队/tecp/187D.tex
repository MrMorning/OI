\newproblem{187D. BRT Contract}
\begin{prob}
	一条路上有$n$个红绿灯,共$n+1$段路。
	起点为左端点,终点为右端点。给出每段路的长度。
	已知所有红绿灯的绿灯周期均为$g$,
	红灯周期均为$r$,且在$[0, g)$时刻(模$g+r$意义下)区间内为绿灯,
	$[g, g+r)$时刻区间内为红灯。\par
	现在有$q$辆公交车从起点出发,开向终点,
	第$i$辆公交出发时刻为$t_i$。求所有公交
	到达终点的时刻。$n \le 10^5$。
\end{prob}

\begin{sol}
	可以发现,一旦公交停在某个红绿灯前,
	下一次出发又是从$0$时刻开始。因此,
	可以预处理一个数组$f$,其中$f[i]$
	表示$0$时刻在第$i$个红绿灯前,到达终点
	的时刻。这个可以通过递推来得到。假设
	已经算出了$k+1 \cdots n$的$f$值。在
	计算$f[k]$时,只需知道从$k$出发第一个
	遇到的红灯是第$t$个,那么$f[k]=$$k$到
	$t$距离$+$等待时间$+f[t]$。利用
	前缀和可以算出从$k$到之后任意位置不停留
	的路程,找到最靠左的落在$[g,g+r)$区间内
	就是停靠的红绿灯。这可以用线段树来实现。
	\par 对于从起点出发的公交车处理方法是
	一样的,同样只需找到第一个停靠位置就能
	推出答案了。
\end{sol}
