\newproblem{123E. Maze}
\begin{prob}
	一个maze由一个树表示。考虑一个如下的子过程:
	\begin{verbatim}
	    DFS(x)
	        if x == exit vertex then
	            finish search
	        flag[x] <- TRUE
	        random shuffle the vertices' order in V(x) 
	        for i <- 1 to length[V] do
	            if flag[V[i]] = FALSE then
	                count++;
	                DFS(y);
	        count++;
	\end{verbatim} 
	\par
	现在给出每个点选为起点和终点的概率,求\verb#count#的期望值。
	点数不超过$10^5$。
\end{prob}

\begin{sol}
	设起点为$s$,终点为$t$,$s$到$t$路径上点数为$l$。
	如果以$s$为根建树,$sz[x]$表示$x$的子树大小(不含$x$)。
	那么$t$子树中的点一定不会被经过。
	而$s->t$对答案的贡献为$l-1$。称$s->t$为主轴,那么
	父亲在主轴上,本身又不在主轴上的点所代表的子树对
	答案的贡献又是什么呢? \par
	注意到,一旦偏离主轴进入某个点$u$,那么一定是$u$子树
	中所有点都要走遍,这些点对答案的贡献是2(一进一出)。
	设$u$的父亲是$p$,$p$主轴上的儿子是$v$。进入$u$的概率等于$p$儿子的所有排列中
	$u$在$v$前面的概率,即$\frac{1}{2}$。于是,
	每个不在主轴上且不在$t$子树中的点对答案的贡献恰好是1。
	设总点数为$n$。那么以$s$为起点,$t$为终点的期望答案
	就是$n-l-sz[t]+l-1 = n-sz[t]-1$,只与$t$的子树大小有关。
	于是乎,在一遍dfs过程中动态修改根,维护所有点sz与作为终点概率的乘积和即可。
\end{sol}
