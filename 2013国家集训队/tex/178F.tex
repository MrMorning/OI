\newproblem{178F. Representative Sampling}
\begin{prob}
	有$n(n \le 2000)$个由26个小写字母组成的字符串,要求从中
	选出$k$个,使该集合的权值最大。设$S$为
	选择的集合,则权值定义如下:
	\begin{displaymath}
		\sum_{i,j \in S} \text{$i$,$j$最长公共前缀的长度}
	\end{displaymath}
\end{prob}

\begin{sol}
	初一看,很容易想到用trie这种前缀处理工具。
	将所有字符串插入trie中后,变成了标记$k$个点,
	每个点的权值定义为$m(m-1)$,$m$是子树中标记点
	的数目。于是可以设计一个状态$O(n^2)$,转移$O(n)$
	的树型dp,总复杂度为$O(n^3)$。走到这里就难以
	继续走下去。只好另辟蹊径。\par
	联想到后缀数组求lcp的方法,可以
	将所有字符串按字典序排序。设计状态$f[l][r][k]$
	表示区间$[l,r]$选$k$个的答案。于是:
	\begin{displaymath}
		f[l][r][k] = \max_{l \le  t \le k}\{ f[l][p][t] + f[p+1][r][k-t] + \text{something}\}
	\end{displaymath}
	注意到$p$取值的随意性,不妨取相邻两字符串LCP最小的那个位置。于是:
	\begin{displaymath}
	f[l][r][k] = \max_{l \le  t \le k}\{ f[l][p][t] + f[p+1][r][k-t] + t(k-t) \cdot lcp[p]\}
\end{displaymath}
有用的区间只有$O(n)$个。复杂度比较难算,考虑两类极端情况:
$p=l+1$,转移是$O(1)$的,总共是$O(n^2)$;
$p=(l+r)/2$,$T(n) = (n/2)^2 + 2T(n/2) = O(n^2 \log n)$。
是一个十分优秀的算法。
\end{sol}
