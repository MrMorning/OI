\question[4]{\begin{large}\textbf{0与1的舞姿} \end{large}}
\vspace{6pt}
\par 对于二进制,想必NX的你们早已有所了解。通俗的讲,二进制就是逢2进位的进位制。
十进制转化到二进制的方法也十分容易,以59为例:
\begin{center}
\begin{tabular}{c @{$ \div $} c @{$ = $} c @{$ \ldots $} c}
$59$ & $2$ & $29$ & $1$ \\
$29$ & $2$ & $14$ & $1$ \\
$14$ & $2$ & $7$ & $0$ \\
$7$ & $2$ & $3$ & $1$ \\
$3$ & $2$ & $1$ & $1$ \\
$1$ & $2$ & $0$ & $1$ \\
\end{tabular}
\end{center}
那么倒着往上写则可得到$(59)_{10} = (111011)_{2}$。\par
定义一个$\mathbf{Z} \times \mathbf{Z} \to \mathbf{Z}$的函数$\oplus$,
对于一位二进制数,有$1 \oplus 1 = 0$,$1 \oplus 0 = 1$,$0 \oplus 1 = 1$,$0 \oplus 0 = 0$。
对于高位二进制数,可以独立每一位处理,位数不等在前面补0。\par
例如:$(10110)_2 \oplus (101)_2 = (10011)_2$。
\par 求:$1 \oplus 2 \oplus 3 \oplus \ldots \oplus 19 \oplus 20$的值。
