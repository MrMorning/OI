\section{314. Shortest Paths}
变态题。参见2008集训队作业。\par
主要思想是将$s \to t$前$k$短路转化为单源$k$短路。\par
先从$t$建一棵最短路径树。那么每个路可以表示为一个非树边的序列。
将非树边建成一个有向图,就成功转化了。
问题在于转化后要求每一步都唯一对应一个非树边序列,
就不能直接在最短路径树上补0权边了。
于是需要知道从每个点到根路径上所有非树边。
为了控制新图中边数,可以对每个点建堆,
在新图中建原堆中的父子边。用可持久化技术可以实现这一点。
保证新图中点数和边数均为$O(m\log m)$级别。
并且由于可持久化的特性,恰好能满足路径的一一对应(这个比较难想)。
于是便成功转化了。
\par 
单源$k$短路的求法:维护一个优先级队列存放当前搜出来的所有路径,
每次选最小的输出,再扩展。由于每个点的度数为$O(\log m)$级别的,
可以看作常数,实测非常快。

