\subsection{[例一]SCOI2011 糖果}
\subsubsection{描述}
{\itshape 幼儿园里有$N$个小朋友,lxhgww老师现在想要给这些小朋友们分配糖果,要求每个小朋友都要分到糖果。但是小朋友们也有嫉妒心,总是会提出一些要求,比如小明不希望小红分到的糖果比他的多,于是在分配糖果的时候,lxhgww需要满足小朋友们的$K$个要求。幼儿园的糖果总是有限的,lxhgww想知道他至少需要准备多少个糖果,才能使得每个小朋友都能够分到糖果,并且满足小朋友们所有的要求。}
\subsubsection{输入格式}
{\itshape 输入的第一行是两个整数$N$,$K$。\par
接下来$K$行,表示这些点需要满足的关系,每行3个数字,$X$,$A$,$B$。\par
如果$X=1$, 表示第$A$个小朋友分到的糖果必须和第$B$个小朋友分到的糖果一样多;\par
如果$X=2$, 表示第$A$个小朋友分到的糖果必须少于第$B$个小朋友分到的糖果;\par
如果$X=3$, 表示第$A$个小朋友分到的糖果必须不少于第$B$个小朋友分到的糖果;\par
如果$X=4$, 表示第$A$个小朋友分到的糖果必须多于第$B$个小朋友分到的糖果;\par
如果$X=5$, 表示第$A$个小朋友分到的糖果必须不多于第$B$个小朋友分到的糖果;}
\subsubsection{输出格式}
{\itshape 输出一行,表示lxhgww老师至少需要准备的糖果数,如果不能满足小朋友们的所有要求,就输出-1。}
\subsubsection{数据范围}
{\itshape
对于$30\%$的数据,保证$N < 100$; \par
对于$100\%$的数据,保证$N < 100000$;\par
对于所有的数据,保证$K \leq 100000$,$ 1 \leq X \leq 5$,$1 \leq A,B \leq N$。
}
\subsubsection{分析}
可以发现,本题就是十分明显的差分约束系统,只不过要最小化$\sum_{i=1}^{n}{x_i}$。为每个未知数建立一个对应顶点,同时为满足隐含的$x_i \geq 1$,建一个附加源$s$,向其他每个点连一条长度为1的边。求SSSP。如果出现负环,则无解,输出-1;否则,$\sum_{i=1}^n{f[i]}$就是答案。
\par 但是,点数和边数都达到了100000,单纯用SPFA很难在规定时限1s内通过所有测试点。
\\[5pt]
\par 
\textbf{优化1:SLF(Small Label First)} 
\par 设队首元素为$h$,队列中要加入节点$u$,在$f[u] \leq f[h]$ 时加到队首而不是队尾,否则和普通的SPFA一样加到队尾。
\par 
\textbf{优化2:LLL(Large Label Last)} 
\par 设队列$Q$中的队首元素$h$,$f$的平均值$\overline{f} = \frac{\sum_{i \in Q} f[i]}{|Q|}$,每次出队时,设出队元素为$u$,若$f[u] > \overline{f}$,把$u$移到队列末尾,如此反复,直到找到一个$u$使$f[u] \leq \overline{f}$ ,将其出队。
\\[4pt] \par
实践证明,加上以上两个优化就可以AC。但渐进时间复杂度并未得到改善。我们尝试寻找特殊,来寻求突破。\par
我们发现,本题的二元约束条件的常数要么为0,要么为-1。如果没有环,就可以直接DFS或者BFS来在线性时间内求最短路。那么,能不能把环强行去掉?考察会出现环的情况,环上的边权只可能是0、-1!如果一个强连通分量(SCC)中的边权不全为0,只能是无解。否则,强连通分量中所有点完全没有区别,我们可以收缩所有SCC,得到一个DAG图,这样就为DP创造了条件。线性时间就能非常完美地解决问题。还需注意的是,要用上64位整型,tarjan求SCC可能要手写栈。
