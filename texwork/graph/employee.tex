\subsection{[例三]NOI2008 志愿者招募}
\subsubsection{描述}
{\itshape
	申奥成功后,布布经过不懈努力,终于成为奥组委下属公司人力资源部门的
		主管。布布刚上任就遇到了一个难题:为即将启动的奥运新项目招募一批短期志
		愿者。经过估算,这个项目需要 N 天才能完成,其中第 i 天至少需要 $A_i$ 个人。
		布布通过了解得知,一共有 M 类志愿者可以招募。其中第 i 类可以从第 $S_i$ 天工
		作到第 $T_i$ 天,招募费用是每人 $C_i$ 元。新官上任三把火,为了出色地完成自己的
		工作,布布希望用尽量少的费用招募足够的志愿者,但这并不是他的特长!于是
		布布找到了你,希望你帮他设计一种最优的招募方案。
}
\subsubsection{输入格式}
{\itshape
	第一行包含两个整数 $N$, $M$,表示完成项目的天数和
		可以招募的志愿者的种类。
		\par 接下来的一行中包含 $N$ 个非负整数,表示每天至少需要的志愿者人数。
		\par 接下来的 $M$ 行中每行包含三个整数 $S_i$, $T_i$, $C_i$,含义如上文所述。为了方便起见,我们可以认为每类志愿者的数量都是无限多的。
}

\subsubsection{输出格式}
{\itshape
	仅包含一个整数,表示你所设计的最优方案的总费
		用。
}
\subsubsection{数据范围}
$30\%$的数据中,$1 \le N, M \le 10$,$1 \le A_i \le 10$;
\par $100\%$的数据中,$1 \le N \le 1000$,$1 \le M \le 10000$,题目中其他所涉及的数据均不超过$2^{31}-1$。
\subsubsection{分析}
设$X_i$为第$i$类志愿者招募的人数。则应满足
\begin{equation} \label{eq:bs}
\forall j \in \mathbf{Z} \cap [1, N], \, \sum_{S_i \le j \le T_i}{X_i} \ge A_i
\end{equation}
\par 添加辅助变量$Y_i$,使\eqref{eq:bs}变为等式
\begin{displaymath}
\sum_{S_i \le j \le T_i}{X_i} - Y_j = A_j
\end{displaymath}
\par 等价于
\begin{displaymath} 
\sum_{S_i \le j \le T_i}{X_i} - Y_j - A_j = 0
\end{displaymath}
\par 似乎又能利用流量守恒建图,但如何满足“左右各一次”,即在一个等式中符号为正,在另一个等式中符号为负?
\par 题目有一个重要的性质没有用到:每类志愿者只管\textbf{连续的一段}时间。这能否为解题提供服务?
\par \textbf{等式整体作差!}如果作差的话,许多变量就会两两相消,只有在连续的一段的两端会留下\textbf{两个}变量!
\par 再看看目标函数
\begin{displaymath}
\text{最小化}\sum_{1 \le i \le M}{X_i \times C_i}
\end{displaymath}
\par 这不就是最小费用流吗?只要在$X_i$对应的边上加权,其余边权值为0,用最小费用最大流即可解决。
\vspace{8pt} \par 本题还存在另一种直接建图的方法,但不如不等式来的清晰,在此就不作介绍。
