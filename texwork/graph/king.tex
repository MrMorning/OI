\subsection{[例四]SGU206 Roads}
\subsubsection{描述}
{\itshape
	一个遥远的王国有$m$条道路连接着$n$个城市$m$条道路中有$n-1$条石头路,其余有
		$m-n+1$ 条泥土路,任意两个城市之间有且仅有一条完全用石头路连接起来的道路。
		每条道路都有一个唯一确定的编号,其中石头路编号为$1, \ldots, n-1$,泥土路编号
		为$n, \ldots, m$。
		\par 每条道路都需要一定的维护费,其中第$i$条道路每年需要$C_i$的费用来维护。
		最近该国国王准备只维护部分道路以节省费用。但是他还是希望人们可以在
		任意两个城市间互达。
		\par 国王需要你提交维护每条道路的费用,
		以便他能让他的大臣来挑选出需要维
			护的道路,使得维护这些道路的费用是最少的。
			尽管国王不知道走石头路和走泥土路的区别,
		但是对于人民来说石头路似乎
			是更好的选择。为了人民的利益,你希望维护的道路是石头路。这样你不得不在
			提交给国王的报告中伪造维护费用。你需要给道路$i$伪造一个费用$D_i$,使得这些
			石头路能够被挑选。
			为了不让国王发现,你需要使得真实值与伪造值的差值和,即$f = \sum_{i=1}^m{|D_i - C_i|}$尽量小。
			\par 国王的大臣当然不是白痴,
		全部由石头路组成的方案如果是一种花费最小的
			方案,那么他会选择这种方案。
			\par
			求出真实值与伪造值的差值和的最小值,以及得到该最小值的方案,即每条
			边的修改后的边权$D_i$。
}
\subsubsection{数据范围}
{
\itshape
	$2 \le n \le 60$,$n-1 \le m \le 400$。
}
\subsubsection{分析}
显然为了得到最小值,应降低石头路的费用而提升泥土路的费用。
设第$i$条道路费用调整了$\delta_i$,由最小生成树的性质可知,每加入一条非树边都会出现一个环,
设加入的泥土路为$r$,而为了使石头路依旧是最小生成树的边,必然有
\begin{displaymath}
C_r + \delta_r \ge \max_{k\text{在环中}}\{C_k - \delta_k\}
\end{displaymath}
\par 如果将上式的$\max$去掉,就变成了对每一条在环上的石头路的限制:
\begin{displaymath}
\bigcup_{k\text{在环中}}\{C_r + \delta_r \ge C_k - \delta_k\}
\end{displaymath}
\par 等价于
\begin{displaymath}
\bigcup_{k\text{在环中}}\{\delta_r + \delta_k \ge C_k - C_r\}
\end{displaymath}
\par 式子右为一常数,目标是最小化$\sum_{i = 1}^m \delta_i$。这就是刚刚提出的问题。
显然这可以用单纯形解,但未免也太大才小用了,较高的时间和编程复杂度也使人望而生畏。\par
让我们在来找找特殊,看看模型是否抽象得足够准确。不难发现,每个不等式的两个变量都是一条树边,一条非树边,
这很自然的让人联想到二分图。再关注下不等式。难道不觉得眼熟吗?求二分图最佳匹配的KM算法!\par
如果把每个$\delta$对应成顶点,那么每个不等式就是一条边,边权即为$C_k - C_r$。
而算法执行结束后每个顶点的可行顶标,就是$\delta$的值,跟KM算法完全一致!
\vspace{5pt} \par 在接触本题前,很多人包括我在内,一直以为KM算法没什么用,速度又没费用流好,完全可以用网络流代替,就没有去学。
很多算法最核心的还是在思想上,广博地学习,能潜移默化地影响我们的思维方式,为我们提供类比的素材,看似没有用的东西可能会带来巨大帮助。
\vspace{5pt}\par \textbf{思考:}如果约束条件为$x_{a_k} + x_{b_k} \le c_k$,并且满足二分图性质,如何最大化$\sum{x_i}$?
\par 其实很简单,只需改成求最小权最佳匹配。

\vspace{5pt}\par \textbf{思考:}如果不再满足二分图性质?
\par 一般图的最佳匹配?好像不是这么回事,KM算法根本不适用了。可能只有老老实实用线性规划的一般方法了。

