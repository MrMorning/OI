\section{三角形不等式的启示 Inspiration of Triangle Inequality}
\subsection{预备知识 Preliminary}
图论中最为经典的例子莫过于最短路了。在这里只讨论单源最短路(SSSP)。
设$f[i]$为目前已知的从源点$s$到$i$顶点的最短路径估计,$f^*[i]$为实际最短路。则显然$f[i] \geq f^*[i]$。

\subsubsection{松弛性质 Relaxation Property}
\par 令$d[u][v] = \left\{ \begin{array}{ll}
	\min\{p.weight\} & \text{存在$u$到$v$的路径$p$} \\
	\infty & \text{$u$到$v$不存在路径}
	\end{array} \right.$

\par 可以得到三角形不等式:
\begin{equation} \label{eq:tri}
\forall u, v \in V, f^*[u] \leq f^*[v] + d[v][u]
\end{equation}

若发现对于$v$节点,有$f[v] > f[u] + d[u][v]$,则可以更新$f[v]$使其更接近目标$f^*[v]$,即让 $f[v] = f[u] + d[u][v]$。这便是松弛操作(relaxation)。而几乎所有的最短路算法都依赖于松弛操作。
\par 若对任何节点都无法松弛,那么有$f = f^*$。反之亦然。证明在此略过。

\subsubsection{几个常用算法简介 Some Known Algorithms}
\begin{center}
\begin{tabular}{|l|l|p{8cm}|}
\hline
算法名称 & 时间复杂度 & 实现原理 \\ \hline
Bellman-Ford & $O(|V||E|)$ & 如果最短路存在,则每个顶点最多经过一次,因此不超过$|V|-1$条边。由最优性原理,只需依次考虑经过边数为$1,2,\ldots,|V|-1$的最短路。对每条边松弛$|V|-1$次即可。\\ \hline
SPFA & $O(k|E|)$ & 此算法为Bellman-Ford的改进。并不是每次松弛都是有价值的。只需每次将$f$值变化了的点加入队列,更新次数降低,实际运行效果非常好。\\ \hline
Dijkstra & $O(|E| + |V| \log |V|)$ & 采用标记永久化技术(亦即贪心),保证每次加入点时已求出最短路。不适用于含负边权的图。\\ \hline
DP(递推) & $O(|V| + |E|)$ & 按拓扑关系递推。只适用于DAG(有向无环)图。\\
\hline
\end{tabular}
\end{center}
\par 从上表可以看出,对于特殊性较高的图,往往可以利用性质,设计出非常高效的算法。

\subsection{差分约束系统 System of Difference Constraints}
差分约束系统是指许多形如$x_{a_k} - x_{b_k} \leq c_k$的不等式组。如果求出一个特解$\{x_i\}$,那么所有$\{x_i + k\}$都是一组解\footnote{但无法生成所有解}。所以关键就在于找到一组特解。\par
如果给约束条件变变形,就成了$x_{a_k} \leq x_{b_k} + c_k$,这不就是\eqref{eq:tri}吗?
\par 给每个$x_i$建立一个对应的顶点,每个约束条件建立一条边,求一遍SSSP,得到的$f = f^*$就是一组解。\par
如果约束条件是$x_i - x_j \geq c_k$呢?其实只要两端同时乘以-1,就转化为了原问题。即便不转化,对比\eqref{eq:tri},我们发现本质是在求最长路,最长路的对偶问题就是边权为负的最短路!\par
还有一些特殊情况需要注意。比如对单个未知数的限制。设一附加源点$s$。由于$f[s] = 0$,所以对$x_i$的约束也可变形为$x_i - x_s \leq c_k$。从$s$直接连边即可解决。
\par 如果边权全为正,可以采用效率更高的Dijkstra算法。
\par 下面,看看差分约束系统在信息学中的应用吧。

\subsection{[例一]SCOI2011 糖果}
\subsubsection{描述}
{\itshape 幼儿园里有$N$个小朋友,lxhgww老师现在想要给这些小朋友们分配糖果,要求每个小朋友都要分到糖果。但是小朋友们也有嫉妒心,总是会提出一些要求,比如小明不希望小红分到的糖果比他的多,于是在分配糖果的时候,lxhgww需要满足小朋友们的$K$个要求。幼儿园的糖果总是有限的,lxhgww想知道他至少需要准备多少个糖果,才能使得每个小朋友都能够分到糖果,并且满足小朋友们所有的要求。}
\subsubsection{输入格式}
{\itshape 输入的第一行是两个整数$N$,$K$。\par
接下来$K$行,表示这些点需要满足的关系,每行3个数字,$X$,$A$,$B$。\par
如果$X=1$, 表示第$A$个小朋友分到的糖果必须和第$B$个小朋友分到的糖果一样多;\par
如果$X=2$, 表示第$A$个小朋友分到的糖果必须少于第$B$个小朋友分到的糖果;\par
如果$X=3$, 表示第$A$个小朋友分到的糖果必须不少于第$B$个小朋友分到的糖果;\par
如果$X=4$, 表示第$A$个小朋友分到的糖果必须多于第$B$个小朋友分到的糖果;\par
如果$X=5$, 表示第$A$个小朋友分到的糖果必须不多于第$B$个小朋友分到的糖果;}
\subsubsection{输出格式}
{\itshape 输出一行,表示lxhgww老师至少需要准备的糖果数,如果不能满足小朋友们的所有要求,就输出-1。}
\subsubsection{数据范围}
{\itshape
对于$30\%$的数据,保证$N < 100$; \par
对于$100\%$的数据,保证$N < 100000$;\par
对于所有的数据,保证$K \leq 100000$,$ 1 \leq X \leq 5$,$1 \leq A,B \leq N$。
}
\subsubsection{分析}
可以发现,本题就是十分明显的差分约束系统,只不过要最小化$\sum_{i=1}^{n}{x_i}$。为每个未知数建立一个对应顶点,同时为满足隐含的$x_i \geq 1$,建一个附加源$s$,向其他每个点连一条长度为1的边。求SSSP。如果出现负环,则无解,输出-1;否则,$\sum_{i=1}^n{f[i]}$就是答案。
\par 但是,点数和边数都达到了100000,单纯用SPFA很难在规定时限1s内通过所有测试点。
\\[5pt]
\par 
\textbf{优化1:SLF(Small Label First)} 
\par 设队首元素为$h$,队列中要加入节点$u$,在$f[u] \leq f[h]$ 时加到队首而不是队尾,否则和普通的SPFA一样加到队尾。
\par 
\textbf{优化2:LLL(Large Label Last)} 
\par 设队列$Q$中的队首元素$h$,$f$的平均值$\overline{f} = \frac{\sum_{i \in Q} f[i]}{|Q|}$,每次出队时,设出队元素为$u$,若$f[u] > \overline{f}$,把$u$移到队列末尾,如此反复,直到找到一个$u$使$f[u] \leq \overline{f}$ ,将其出队。
\\[4pt] \par
实践证明,加上以上两个优化就可以AC。但渐进时间复杂度并未得到改善。我们尝试寻找特殊,来寻求突破。\par
我们发现,本题的二元约束条件的常数要么为0,要么为-1。如果没有环,就可以直接DFS或者BFS来在线性时间内求最短路。那么,能不能把环强行去掉?考察会出现环的情况,环上的边权只可能是0、-1!如果一个强连通分量(SCC)中的边权不全为0,只能是无解。否则,强连通分量中所有点完全没有区别,我们可以收缩所有SCC,得到一个DAG图,这样就为DP创造了条件。线性时间就能非常完美地解决问题。还需注意的是,要用上64位整型,tarjan求SCC可能要手写栈。


\subsection{一些思考 Some Thoughts}
如果约束条件未知数的系数不再是0、1,而是任意非负实数呢?即求解不等式组$\{p_k x_{a_k} - q_k x_{b_k} \leq c_k \, | p_k,q_k \ge 0\}$。\par
先做恒等变换,等价于求解$\{x_{a_k} \leq \frac{q_k}{p_k}x_{b_k} + \frac{c_k}{p_k}\}$,如果用SPFA动态更新,直到无法获得更优解,就和最短路没什么两样了。只是复杂度不好估计。\par
如果两元关系不再是作差,而是之和,又如何在满足约束条件的情况下最小化变元之和?如果不再是两元,而是更多,又怎么办?最高次不再是1,单纯形法\footnote{一种解决线性规划的通法}也失效,图能辅助解决吗?这就是接下来所要探索的内容。
