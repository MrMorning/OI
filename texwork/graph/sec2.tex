\section{网络流的守恒思想 The Conservation of Network Flow}
\subsection{预备知识 Preliminary}
\textbf{流网络(flow network)}是一个有向图$G = (V, E)$,其中每条$(u, v) \in E$有一个非负容量上界$c(u, v) \ge 0$。规定:若$(u, v) \notin E, \, c(u, v) = 0$。在有下界的网络流中还存在容量下界$b(u, v)$,与容量上界类似。网络中有两个特殊点:源$s$与汇$t$。
\subsubsection{流的性质 Properties of Flow}
\par 流网络$G$的\textbf{流(flow)}是一个函数$f : V \times V \to \mathbf{R}$,且满足下列三个性质:
\begin{enumerate}
\item 容量限制 \quad $\forall u, v \in V$,$f(u, v) \leq c(u, v)$。在有上下界的网络流中,还满足$\forall u, v \in V$,$b(u, v) \le f(u, v) \leq c(u, v)$
\item 反对称性 \quad $\forall u, v \in V$,$f(u, v) = -f(v, u)$。
\item 流守恒性 \quad $\forall u \in V - \{s, t\}$,$\sum_{v \in V}{f(u, v)} = 0$。
\end{enumerate}
\par 称$f(u, v)$为从$u$到$v$的流。
\par 流$f$定义为$f = \sum_{u \in V} f(s, u)$。
最大流$f_{\text{max}}$定义为$\max\{f\}$。
\subsubsection{求最大流的算法 Algorithms for Maximum Flow Problem}
\par 求无下界最大流常用连续最短增广路算法(Dinic's Algorithm),复杂度为$O(|V|^2|E|)$,实际效果良好,特别地,在二分图中复杂度更低,为$O(\sqrt{|V|}|E|)$。详见\cite{wxs2007}。
\par 若存在下界,往往采用参数搜索,即二分答案后,转化为无源汇的网络流,再结合最大流算法求解。具体参见\cite{zy2004}。
\subsubsection{费用流 Minimum Cost Flow}
若给每条边加上一个权值$w(u, v)$,定义流的费用为$g = \sum_{u, v \in V}{f(u, v) \times w(u, v)}$,其中$f(u, v) > 0$。
\par 类似定义最小费用流的费用$g_{\text{min}}$,最大费用流的费用$g_{\text{max}}$。
\par 求最小费用流简单快速的办法就是最短路增广,这样可以保证每次增广的都是最小费用的路径。由于含负权,采用SPFA。复杂度的界比较高,但实际上远小于复杂度上界。

\subsection{[例二]CTSC2005 玩具的重量}
\subsubsection{描述}
{\itshape 
	冰冰有三个玩具:皮卡丘、维妮孙悟空和芭比娃娃。她并不知道这些玩具的具体重量(采用NOI单位),但是知道每个玩具重量的大概范围,如下表:

		\begin{center}
		\begin{tabular}{|c|c|c|}
	\hline
	玩具名称 & 最小可能重量 & 最大可能重量 \\ \hline
	皮卡丘 & 1 & 3 \\ \hline
	维妮孙悟空 & 2 & 4 \\ \hline
	芭比娃娃 & 3 & 5 \\ \hline
	\end{tabular}
	\end{center}
\par 这些范围太粗略,冰冰希望能把它们缩小一些。 正好佳佳有一个电子天平,不仅可以告诉你左右两边是否一样重,还可以告诉你左边比右边重(或轻)多少。
天平很大,左右两边都可以放任意多件玩具。 冰冰向佳佳借电子天平,希望能算出每个玩具的精确重量。佳佳为了考验冰冰,只允许她把任意一个玩具往天平的左侧和右侧最多各放一次。例如,如果她曾经把皮卡丘放在天平的左侧,则她不能再次把它放在天平的左侧。冰冰同意了。
她一共称量了两次,称量结果如下:
\par 第一次:左边放皮卡丘,右边放维妮孙悟空,返回右边比左边重1;
\par 第二次:左边放维妮孙悟空,右边放芭比娃娃,返回左边比右边重1;
\par 可以确定三个玩具的重量一定是3,4,3,也就是说,通过称量结果所得到的更新后的重量范围是:
		\begin{center}
		\begin{tabular}{|c|c|c|}
	\hline
	玩具名称 & 最小可能重量 & 最大可能重量 \\ \hline
	皮卡丘 & 3 & 3 \\ \hline
	维妮孙悟空 & 4 & 4 \\ \hline
	芭比娃娃 & 3 & 3 \\ \hline
	\end{tabular}
	\end{center}

根据称量结果所得到的精确范围 冰冰以后还会买很多很多玩具,她不想每次都自己计算每个玩具的重量。她需要写一个程序计算每个玩具最精确的重量下限和上限,你能帮她吗?
}
\subsubsection{输入格式}
{\itshape 
	输入文件第一行包含两个整数$n$和$m$,即玩具的个数和称量的次数。第二行包含$2n$个数,第$2i-1$个数和第$2i$个数分别表示第$i$个玩具的重量初始下限和初始上限。以下$m$行,每行前三个数$L$,$R$,$D$表示左边的玩具数、右边的玩具数和左右两边的重量差($L,R \ge 0$),接下来的$L$个数为天平左边的玩具编号,再接下来的$R$个数为天平右边的玩具编号。输入保证每个玩具在天平的每一边最多出现一次。
}
\subsubsection{输出格式}
{\itshape 输出包含$2n$个整数,第$2i-1$个数和第$2i$个数分别表示第$i$个玩具的重量下限和上限,即最小可能的整数重量和最大可能的整数重量。如果无解(可能是天平坏了),只输出一个数$-1$。}
\subsubsection{数据范围}
{\itshape
对于$50\%$的数据,满足$3 \le n \le 10$,$1 \le m \le 5$ ;\par
对于$100\%$的数据,满足$3 \le n \le 2000$,$1 \le m \le 100$,重量上限不超过20000。
}
\subsubsection{分析}
“每个玩具在天平的每一边最多出现一次”这个条件十分诱人,而每次称量结果为一等式,恰好能对应上流守恒性。
我们将每次称量作为一个点,由于最多出现一次,玩具可作为边。
\begin{itemize}
\item 如果左右两边各出现一次,则在对应称量的两点间连一条边。
\item 如果只在左边出现一次,则从源向该点连一条边。
\item 如果只在右边出现一次,则从该点向汇连一条边。
\item 如果没有出现,则从源向汇连一条边。
\end{itemize}
\par 边的容量上下限均取玩具重量的上下界。那么对应的流量就是玩具的重量。每一个可行流就是一组合法解。
\par 那么,如何确定每个物品的解区间?如果我们强制是一条边的容量上界等于容量下界,就等于强制压流。
而在已知一组解的情况下,解区间被划成了两块单调的区间。设0表示非法,1表示合法,那么必然是
\begin{displaymath}
0000\ldots000011111\ldots11111\ldots111100000\ldots000
\end{displaymath}
\par 左右两边分别二分判可行性即可。


\vspace{10pt} \par 根据等式联想到流量平衡,由上下界网络流强制限流的特点来判可行性,实在是巧妙。即便是不等式,也能通过添加辅助变量,来变成等式。“左右两边最多各出现一次”也是不可或缺的条件,通过简单的变形,往往也能转化成这个条件,比如说下面这道题。

\subsection{[例三]NOI2008 志愿者招募}
\subsubsection{描述}
{\itshape
	申奥成功后,布布经过不懈努力,终于成为奥组委下属公司人力资源部门的
		主管。布布刚上任就遇到了一个难题:为即将启动的奥运新项目招募一批短期志
		愿者。经过估算,这个项目需要 N 天才能完成,其中第 i 天至少需要 $A_i$ 个人。
		布布通过了解得知,一共有 M 类志愿者可以招募。其中第 i 类可以从第 $S_i$ 天工
		作到第 $T_i$ 天,招募费用是每人 $C_i$ 元。新官上任三把火,为了出色地完成自己的
		工作,布布希望用尽量少的费用招募足够的志愿者,但这并不是他的特长!于是
		布布找到了你,希望你帮他设计一种最优的招募方案。
}
\subsubsection{输入格式}
{\itshape
	第一行包含两个整数 $N$, $M$,表示完成项目的天数和
		可以招募的志愿者的种类。
		\par 接下来的一行中包含 $N$ 个非负整数,表示每天至少需要的志愿者人数。
		\par 接下来的 $M$ 行中每行包含三个整数 $S_i$, $T_i$, $C_i$,含义如上文所述。为了方便起见,我们可以认为每类志愿者的数量都是无限多的。
}

\subsubsection{输出格式}
{\itshape
	仅包含一个整数,表示你所设计的最优方案的总费
		用。
}
\subsubsection{数据范围}
$30\%$的数据中,$1 \le N, M \le 10$,$1 \le A_i \le 10$;
\par $100\%$的数据中,$1 \le N \le 1000$,$1 \le M \le 10000$,题目中其他所涉及的数据均不超过$2^{31}-1$。
\subsubsection{分析}
设$X_i$为第$i$类志愿者招募的人数。则应满足
\begin{equation} \label{eq:bs}
\forall j \in \mathbf{Z} \cap [1, N], \, \sum_{S_i \le j \le T_i}{X_i} \ge A_i
\end{equation}
\par 添加辅助变量$Y_i$,使\eqref{eq:bs}变为等式
\begin{displaymath}
\sum_{S_i \le j \le T_i}{X_i} - Y_j = A_j
\end{displaymath}
\par 等价于
\begin{displaymath} 
\sum_{S_i \le j \le T_i}{X_i} - Y_j - A_j = 0
\end{displaymath}
\par 似乎又能利用流量守恒建图,但如何满足“左右各一次”,即在一个等式中符号为正,在另一个等式中符号为负?
\par 题目有一个重要的性质没有用到:每类志愿者只管\textbf{连续的一段}时间。这能否为解题提供服务?
\par \textbf{等式整体作差!}如果作差的话,许多变量就会两两相消,只有在连续的一段的两端会留下\textbf{两个}变量!
\par 再看看目标函数
\begin{displaymath}
\text{最小化}\sum_{1 \le i \le M}{X_i \times C_i}
\end{displaymath}
\par 这不就是最小费用流吗?只要在$X_i$对应的边上加权,其余边权值为0,用最小费用最大流即可解决。
\vspace{8pt} \par 本题还存在另一种直接建图的方法,但不如不等式来的清晰,在此就不作介绍。

