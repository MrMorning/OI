\subsection{[例二]CTSC2005 玩具的重量}
\subsubsection{描述}
{\itshape 
	冰冰有三个玩具:皮卡丘、维妮孙悟空和芭比娃娃。她并不知道这些玩具的具体重量(采用NOI单位),但是知道每个玩具重量的大概范围,如下表:

		\begin{center}
		\begin{tabular}{|c|c|c|}
	\hline
	玩具名称 & 最小可能重量 & 最大可能重量 \\ \hline
	皮卡丘 & 1 & 3 \\ \hline
	维妮孙悟空 & 2 & 4 \\ \hline
	芭比娃娃 & 3 & 5 \\ \hline
	\end{tabular}
	\end{center}
\par 这些范围太粗略,冰冰希望能把它们缩小一些。 正好佳佳有一个电子天平,不仅可以告诉你左右两边是否一样重,还可以告诉你左边比右边重(或轻)多少。
天平很大,左右两边都可以放任意多件玩具。 冰冰向佳佳借电子天平,希望能算出每个玩具的精确重量。佳佳为了考验冰冰,只允许她把任意一个玩具往天平的左侧和右侧最多各放一次。例如,如果她曾经把皮卡丘放在天平的左侧,则她不能再次把它放在天平的左侧。冰冰同意了。
她一共称量了两次,称量结果如下:
\par 第一次:左边放皮卡丘,右边放维妮孙悟空,返回右边比左边重1;
\par 第二次:左边放维妮孙悟空,右边放芭比娃娃,返回左边比右边重1;
\par 可以确定三个玩具的重量一定是3,4,3,也就是说,通过称量结果所得到的更新后的重量范围是:
		\begin{center}
		\begin{tabular}{|c|c|c|}
	\hline
	玩具名称 & 最小可能重量 & 最大可能重量 \\ \hline
	皮卡丘 & 3 & 3 \\ \hline
	维妮孙悟空 & 4 & 4 \\ \hline
	芭比娃娃 & 3 & 3 \\ \hline
	\end{tabular}
	\end{center}

根据称量结果所得到的精确范围 冰冰以后还会买很多很多玩具,她不想每次都自己计算每个玩具的重量。她需要写一个程序计算每个玩具最精确的重量下限和上限,你能帮她吗?
}
\subsubsection{输入格式}
{\itshape 
	输入文件第一行包含两个整数$n$和$m$,即玩具的个数和称量的次数。第二行包含$2n$个数,第$2i-1$个数和第$2i$个数分别表示第$i$个玩具的重量初始下限和初始上限。以下$m$行,每行前三个数$L$,$R$,$D$表示左边的玩具数、右边的玩具数和左右两边的重量差($L,R \ge 0$),接下来的$L$个数为天平左边的玩具编号,再接下来的$R$个数为天平右边的玩具编号。输入保证每个玩具在天平的每一边最多出现一次。
}
\subsubsection{输出格式}
{\itshape 输出包含$2n$个整数,第$2i-1$个数和第$2i$个数分别表示第$i$个玩具的重量下限和上限,即最小可能的整数重量和最大可能的整数重量。如果无解(可能是天平坏了),只输出一个数$-1$。}
\subsubsection{数据范围}
{\itshape
对于$50\%$的数据,满足$3 \le n \le 10$,$1 \le m \le 5$ ;\par
对于$100\%$的数据,满足$3 \le n \le 2000$,$1 \le m \le 100$,重量上限不超过20000。
}
\subsubsection{分析}
“每个玩具在天平的每一边最多出现一次”这个条件十分诱人,而每次称量结果为一等式,恰好能对应上流守恒性。
我们将每次称量作为一个点,由于最多出现一次,玩具可作为边。
\begin{itemize}
\item 如果左右两边各出现一次,则在对应称量的两点间连一条边。
\item 如果只在左边出现一次,则从源向该点连一条边。
\item 如果只在右边出现一次,则从该点向汇连一条边。
\item 如果没有出现,则从源向汇连一条边。
\end{itemize}
\par 边的容量上下限均取玩具重量的上下界。那么对应的流量就是玩具的重量。每一个可行流就是一组合法解。
\par 那么,如何确定每个物品的解区间?如果我们强制是一条边的容量上界等于容量下界,就等于强制压流。
而在已知一组解的情况下,解区间被划成了两块单调的区间。设0表示非法,1表示合法,那么必然是
\begin{displaymath}
0000\ldots000011111\ldots11111\ldots111100000\ldots000
\end{displaymath}
\par 左右两边分别二分判可行性即可。
