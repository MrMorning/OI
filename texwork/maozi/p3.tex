\section{第三题}
当 $ n \le 14 $ 时 \par
$ \frac{\sqrt{(n+1)!}}{3^n} \le \sqrt{\frac{15!}{(3^{14})^2}} = \sqrt{\frac{15 \times 14 \times \cdots \times 2}{9 \times 9 \times \cdots \times 9}} $ \par
$ \because \frac{(9+i)(9-i)}{9 \cdot 9} = \frac{9^2-i^2}{9^2} < 1, i = 1,2,\cdots,6 $ \par
$ \therefore \sqrt{\frac{15!}{9^14}} < \sqrt{\frac{2}{9}} < 1, \therefore \sqrt{\frac{(n+1)!}{(3^n)^2}} < 1 $ \par
$ \because \lvert P_0P_i \rvert \ge d, i = 1,2,\cdots,n. $ \par
$ \therefore \prod_{i=1}^{n} \lvert P_0P_i \rvert \ge d^n > d^n\frac{\sqrt{(n+1)!}}{3^n} = (\frac{d}{3})^n\sqrt{(n+1)!} $ \par
当 $ n > 14 $ 时 \par
设 $ n = k $ 时命题成立. \par
当 $ n = k + 1 $ 时,下面用反证法.证明:$\lvert P_0P_1 \rvert,\cdots,\lvert P_0P_{n+1}$中,存在一个大于等于$\frac{d}{3}\sqrt{n+2}$. \par
否则,假设不成立.则以$P_0$位圆心,$d(\frac{\sqrt{n+2}}{3}+\frac{1}{2})$为半径作圆,再以$P_i(i=1,2,\cdots,n+1)$为圆心,$\frac{d}{2}$为半径作圆.则$ \odot P_i(i=1,2,\cdots,n+1) $均内含于$\odot P_0(d(\frac{\sqrt{n+2}}{3}+\frac{1}{2})>\frac{d}{2}+\frac{d}{3}\sqrt{n+2}>\frac{d}{2}+\lvert P_0P_i \rvert )$. \par
且$\odot P_i(i=1,2,\cdots,n+1)$均外离或外切(否则,设$\odot P_i$与$\odot P_j$交于$A$,$B$两点,则$P_i,A,P_j$不共线,$ \lvert P_iP_j \rvert < \lvert P_iA \rvert + \lvert P_jA \rvert = d. $ 矛盾). \par
考虑它们的面积则 $ \pi d^2(\frac{\sqrt{n+2}}{3}+\frac{1}{2})^2>\pi\frac{d^2}{4}(n+1) $ \par
$ \therefore \frac{n+2}{9} + \frac{1}{4} + \frac{\sqrt{n + 2}}{3} > \frac{n+1}{4},12\sqrt{n+2}>5n-8,144n+288>25n^2-80n+64.25n^2-224n-224<0. $ \par
设$f(x)=25x^2-224x-224, \therefore f(x) $ 的对称轴为$x=\frac{112}{25}<14$. \par
又$f(14)=126 \times 14 -224 > 0. \therefore n < 14$与$n \ge 14$矛盾. \par
$ \therefore$ 不妨设$\lvert P_0P_{n+1} \rvert = \frac{d}{3}\sqrt{n+2}$. \par
由归纳假设,因为$P_0,P_1,\cdots,P_n$之间距离大于等于$d$. $\therefore \prod_{i=1}{n}\lvert P_0P_i \rvert > (\frac{d}{3})^n\sqrt{(n+1)!} $. \par
$ \therefore \prod_{i=1}^{n+1} \lvert P_0P_i \rvert > (\frac{d}{3})^n\sqrt{(n+1)!} \cdot \frac{d}{3}\sqrt{n+2} = (\frac{d}{3})^(n+1)\sqrt{(n+2)!}. $ \par
由$n=k+1$时,命题成立. \par
由数学归纳法,命题对$n \in N^{*}$成立即$\prod_{i=1}^{n} \lvert P_0P_i \rvert > (\frac{d}{3})^n\sqrt{(n+1)!} $.
