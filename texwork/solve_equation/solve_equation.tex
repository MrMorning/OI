\documentclass{article}
\usepackage{fontspec}
\usepackage{indentfirst}
\usepackage{listings}
\usepackage[colorlinks,linkcolor=blue,anchorcolor=blue,citecolor=green]{hyperref}

\setmainfont{文泉驿微米黑}
\XeTeXlinebreaklocale &quot;zh&quot;
\XeTeXlinebreakskip = 0pt plus 1pt minus 0.1pt
\title{求解$x^a \equiv b(mod\ m)$}
\author{CDQZ OI Team\footnote{发现者:before\_rain,探索者:ymfoi,执笔者:csimstu}}
\begin{document}
\maketitle
\tableofcontents

\section{题目描述}
求方程$x^a \equiv b(mod\ m)$的所有解,其中$m$是素数。
\section{理论支持}
\subsection{定义} $r$对模$m$的指数$Ord_m(r)$为使$r^d \equiv 1\ (mod\ m)$成立的最小正整数,若$Ord_m(r) = \varphi(m)$,则称$r$是$m$的原根。
\subsection{性质} 令$\delta = Ord_m(r)$,则$r^0,r^1,r^2 \ldots ,r^{\delta-1}$模$m$两两不同余。假设同余,就可以得到一个更小的$\delta$,于假设不符。
\subsection{用途} 由于$m$是素数,由费马小定理知,$r^{m-1} \equiv 1 \ (mod\ m)$,故$\delta = \varphi(m) = m - 1$。于是,$r^0,r^1,r^2 \ldots ,r^{m-2}$恰好组成了一个$1$到$m-1$的排列\footnote{只考虑$r > 0$的情况,$r = 0$可以特判。}。反过来,每一个$1$到$m-1$的的数(即这个同余系中的每一个数)都可以表成$r^k$。不妨设$x = r^i$,$b = r^j$,于是有
\begin{center}
$r^{ia} \equiv r^j\ (mod\ m)$
\end{center}
因为$r^0,r^1,r^2 \ldots ,r^{m-2}$恰好组成了一个$1$到$m-1$的排列,所以上式可写成$ia \equiv j\ (mod\ m - 1)$。如果求得了$j$,就能解出$i$,从而得到$x=r^i$的值。
\section{Stage I: 求$m$的原根}
目前还没有快速求原根的方法。幸运的是,解决本问题只需求出一个原根,且经过科学验证,在小于$10^9$的数中,每个数最小的原根最多只有几十。枚举原根$r$即可。验证也可以做得很快:可以证明,只需验证$\frac{m-1}{p}$,其中$p$是能整除$m-1$的素数。可以暴力分解质因数。
\section{Stage II: 求离散对数$j$}
因为$j \leq m-1$,可以设$p = \lfloor \sqrt{m-1} \rfloor$,这样一来,$j$可以写成$j = pk + s$,只需要枚举$k$于$s$中的一个,利用乘法逆元,看另一个是否能满足。如果套上map,复杂度为$O(\sqrt{m} \log m)$;
\section{Stage III: 解线性同余方程}
直接上扩展欧几里得算法。在算每一个解的时候,可以暴力枚举,直到出现重复,因为答案肯定不会太多。
\section{注意事项}
\begin{enumerate}
\item 不管是乘法还是加法都要记得转long long并取模。特别注意加法以及负数。
\item 用扩展欧几里得算法时要注意答案是否完整。
\end{enumerate}

\section{代码}
\begin{enumerate}
\item \href{http://www.tedyin.com/blog/archive/567/}{ymfoi的代码},特点:形状优美,结构匀称。
\item \href{https://ideone.com/5Uuhi}{csimstu的代码},特点:耦合度低,易于阅读,STL使用较多,pascaler慎入。
\end{enumerate}
\end{document}
