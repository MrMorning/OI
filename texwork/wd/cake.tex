\section{蛋糕啃王迪}
\subsection{题目描述}
{\itshape
	昨日是王迪16大寿,于是他决定晚上在科技楼4楼过道摆寿宴。为了适合过道的形状,王迪做了$K$($K \le 10^5$)条非常长的油条状的蛋糕,第$i$条蛋糕被平均切成了$N_i$($Ni \le 10^9$)块。由于Hobo和csimstu通知不到位,只来了ZWJ一个人。\par
		宴会开始了,王迪和ZWJ从第1条开始轮流吃蛋糕。为了不让桌子显得太乱,吃每条蛋糕的时候,他们只会从蛋糕的一个端点拿蛋糕。但是王迪做的这$K$条蛋糕的口味不相同,有的口味轻,有的口味重。在吃第$i$块蛋糕时,轮到其中一人吃的时候,出于礼貌,至少要拿一块来吃,至多能拿$M_i$($0 < M_i$)块蛋糕吃。因为王迪是大家公认的好男人,王迪当然ZWJ先吃蛋糕了。 \par
		做蛋糕也是件考验人品的事,每条蛋糕总有那么一块是最好吃的,Hobo和csimstu在宴会开始前,偷偷地将每一块蛋糕都咬了一口,得知第$i$条蛋糕的第$A_i$块蛋糕是这条蛋糕中最好吃的一块。
		你能告诉Hobo和csimstu每一条蛋糕最好吃的那块一定能被谁吃掉吗?
		(假设王迪和ZWJ的食量无限大,把所有蛋糕吃完了)
}
\subsection{输入格式}
{\itshape
第一行为整数$K$。
\par 接下来$K$行,每行三个整数:蛋糕被切成的块数$N_i$,
和这块蛋糕一次至多能拿$M_i$,以及$A_i$,最好吃的一块。
}
\subsection{输出格式}
{\itshape 共$K$行,每一行为“ZWJ”或“王迪”。}
\subsection{样例输入}
\begin{verbatim}
2
5 5 1
30 20 4
\end{verbatim}
\subsection{样例输出}
\begin{verbatim}
ZWJ
王迪
\end{verbatim}
