\section{妹子的礼物}
\subsection{题目描述}
{\itshape 王迪有着非常好的妹子缘,所以每当生日来到的时候,一大乐事就是收到一大堆妹子的礼物。但妹子都比较畏惧LYD先生,同时为了能获得独自与王迪交流的时间,希望王迪主动到妹子的班级领取,而不是把礼物送到王迪的班上来。
\par 为了简化问题,可以把王迪和妹子看作三维直角坐标系中的点,其中任意两点之间没有障碍(都能互相达到)。王迪当然不会错过任何一个妹子的礼物,并且他打算只见每一个妹子一面,最后回到自己的教室。但课间时间紧迫,他只能选择一条最短的路来走。同时由于王迪急切地想见妹子,编程功力退步,希望你能帮他解决这个问题。
}
\subsection{输入格式}
{\itshape
第一行包括三个实数$x_0$,$y_0$,$z_0$,为王迪最初的位置。
\par 第二行为一个数$n$($0 \le n \le 100$),为妹子个数。
\par 接下来$n$行,第$i$行包括三个实数$x_i$,$y_i$,$z_i$,为第$i$个妹子的坐标。}
\subsection{输出格式}
{\itshape
第一行包含一个实数,即最短的总路程,保留$3$位小数。
\par 第二行包含$n$个数,即依次访问的妹子编号。}
\subsection{样例输入}
\begin{verbatim}
	0 0 0 
	1 
	1 1 1 
\end{verbatim}
\subsection{样例输出}
\begin{verbatim}
	1.732
	1
\end{verbatim}

\subsection{数据规模}
{\itshape
对于$30\%$的数据,$n \le 10$;\par
对于$60\%$的数据,$n \le 20$;\par
对于$100\%$的数据,$n \le 100$,所有实数均大于$-10^9$小于$10^9$。
}
\subsection{评分标准}
{\itshape
	如果输出的方案与答案不吻合,则该测试点得$0$分。\par
	否则设提供的答案为$y$,你的答案为$x$,得分为$score$,那么
	\begin{displaymath}
	score = \left\{ \begin{array}{ll}
		10 & x > y \\
		10^{\frac{x}{y}} & x \le y \end{array}
		\right.
	\end{displaymath}
}
